\documentclass{article}
\usepackage[utf8]{inputenc}

\title{\textbf{Analyzing Fixed Point Arithmetic Rounding Error}}
\author{Transmissions11}
\date{February 2022}

\begin{document}

\maketitle

\begin{abstract}

We examine the magnitude of rounding error resulting from different combinations of fixed point arithmetic operations used to achieve the same goal, in an environment with truncating integer semantics.

\end{abstract}

\tableofcontents

\section{Definitions}

\subsubsection{Epsilon Notation}

By definition of the floor operation, the following is true:

$$ \frac{x}{y} - 1 < \left\lfloor\frac{x}{y}\right\rfloor \leq \frac{x}{y} $$

To simplify the inequality above for use in larger expressions, we will define epsilon ($\epsilon$) as a non-deterministic number in the interval $[0, 1)$. Using this new notation, we can rewrite the inequality above like so:

$$ \frac{x}{y} - \epsilon = \left\lfloor\frac{x}{y}\right\rfloor $$

\section{Analysis}

A common use case for fixed point arithmetic is multiplying a quantity of one asset by a conversion rate to another asset, in a programming language with no floating point arithmetic, only integer operations. A notable example of a language with these properties is the Solidity smart contract language, which only has basic truncating integer operations.

\subsection{Introduction}

Let $\sigma_n$ be an arbitrary quantity of asset A that we wish to exchange for a quantity of asset B, $\mu_n$. The rate of exchange between asset A and asset B is defined as the ratio between the separate quantities of $\mu$ and $\sigma$. 

Below we will define and analyze multiple implementations of the desired function, $\mbox{to}_\mu(\sigma_n)$, which takes a quantity of asset A ($\sigma_n$) and returns the amount of asset B that the quantity of asset A is worth ($\mu_n$) using the rate of exchange formula described briefly above ($\frac{\mu}{\sigma}$).

\subsection{Fixed Loss Implementation}

In an environment without truncating division, we can observe that these two implementations are equivalent:

$$ \mu_n = \mbox{to}_\mu(\sigma_n) = \sigma_n \cdot \frac{\mu}{\sigma} = \frac{\sigma_n\mu}{\sigma} $$

However, in an environment with truncating division, these implementations differ:

$$ \mu_n = \mbox{to}_\mu(\sigma_n) = \sigma_n \cdot \left\lfloor\frac{\mu}{\sigma}\right\rfloor \leq \left\lfloor\frac{\sigma_n\mu}{\sigma}\right\rfloor $$

Rewriting using the epsilon notation introduced earlier:

$$ \mu_n = \mbox{to}_\mu(\sigma_n) = \sigma_n \cdot \left(\frac{\mu}{\sigma} - \epsilon\right) \leq \frac{\sigma_n\mu}{\sigma} - \epsilon $$

Distributing multiplication:

$$ \mu_n = \mbox{to}_\mu(\sigma_n) = \frac{\mu}{\sigma}\sigma_n - \epsilon\sigma_n \leq \frac{\sigma_n\mu}{\sigma} - \epsilon $$


Now, with the expressions fully expanded, we can easily observe that the expression where multiplication is performed before division is a much better approximation of the pure, non-truncating exchange rate formula.

Compared to the scaled rounding loss in the left equation ($\epsilon\sigma_n$), where rounding error is unbounded, rounding error in the equation on the right is bounded between $[0, 1)$.

\newpage

\subsection{Fixed Point Arithmetic Implementations}

\subsubsection{$\sigma_n.fmul(\mu.fdiv(\sigma))$}

$$ \mu_n = \mbox{to}_\mu(\sigma_n) = \left\lfloor\frac{\left\lfloor\frac{\mu \cdot 10^{18}}{\sigma}\right\rfloor \cdot \sigma_n}{10^{18}}\right\rfloor $$

Using epsilon notation:

$$ \mu_n = \mbox{to}_\mu(\sigma_n) = \frac{(\frac{\mu \cdot 10^{18}}{\sigma} - \epsilon) \cdot \sigma_n}{10^{18}} - \epsilon $$

Simplified:

$$ \mu_n = \mbox{to}_\mu(\sigma_n) = \frac{\mu\sigma_n}{\sigma} - \frac{\epsilon\sigma_n}{10^{18}} - \epsilon $$

Error scales proportionally with $\sigma_n$.

\subsubsection{$\sigma_n.fmul(\mu).fdiv(\sigma)$}

$$ \mu_n = \mbox{to}_\mu(\sigma_n) = \left\lfloor\frac{\left\lfloor\frac{\sigma_n \cdot \mu}{10^{18}}\right\rfloor \cdot 10^{18}}{\sigma}\right\rfloor $$

Epsilon notation:

$$ \mu_n = \mbox{to}_\mu(\sigma_n) = \frac{\left(\frac{\sigma_n \cdot \mu}{10^{18}} - \epsilon\right) \cdot 10^{18}}{\sigma} - \epsilon $$

Simplified:

$$ \mu_n = \mbox{to}_\mu(\sigma_n) = \frac{\mu\sigma_n}{\sigma} - \frac{\epsilon10^{18}}{\sigma} - \epsilon$$

Error is inversely proportional with $\sigma$.

\subsubsection{$\sigma_n.fdiv(\sigma).fmul(\mu)$}

$$ \mu_n = \mbox{to}_\mu(\sigma_n) = \left\lfloor\frac{\left\lfloor\frac{\sigma_n \cdot 10^{18}}{\sigma}\right\rfloor \cdot \mu}{10^{18}}\right\rfloor $$

Epsilon notation:

$$ \mu_n = \mbox{to}_\mu(\sigma_n) = \frac{\left(\frac{\sigma_n \cdot 10^{18}}{\sigma} - \epsilon\right) \cdot \mu}{10^{18}} - \epsilon $$

Simplified: 

$$ \mu_n = \mbox{to}_\mu(\sigma_n) = \frac{\mu\sigma_n}{\sigma} - \frac{\epsilon\mu}{10^{18}} - \epsilon $$

Error scales proportionally with $\mu$.

\end{document}
